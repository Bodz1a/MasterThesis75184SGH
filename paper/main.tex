% Style for a MSc paper at Warsaw School of Economics
% Michał Ramsza
% Fri Dec 22 14:49:36 CET 2023

% --- document class and other global stuff ---------------------------
\documentclass[english, twoside, 12pt, a4paper]{article}

% --- packages --------------------------------------------------------
\usepackage{textcomp}
\usepackage{times}
\usepackage{amsmath}
\usepackage{amsfonts}
\usepackage{amssymb}
\usepackage{amsthm}
\usepackage[T1]{fontenc}
\usepackage[utf8]{inputenc}
\usepackage{graphicx}
\usepackage{tikz}
\usepackage{xcolor}
\usepackage{enumitem}
\usepackage[english]{babel}
\usepackage{booktabs}
\usepackage{setspace}
\usepackage{csquotes}
\usepackage[centering, left=3.5cm, right=2.5cm, textheight=24cm]{geometry}

% --- packages for citations ------------------------------------------
\usepackage[backend=biber, style=authoryear, autocite=inline, defernumbers=true]{biblatex}
\addbibresource{refs.bib}

% --- package for automatic insertion of R code -----------------------
\usepackage{listings}
\lstset{%
   numbers=left,%
   tabsize=3,%
   numberstyle=\footnotesize,%
   basicstyle=\ttfamily \small \color{black},%
   keywordstyle=\ttfamily \small \color{black},%
   commentstyle=\ttfamily \small \color{gray},%
   stringstyle=\ttfamily \small \color{black},%
   identifierstyle=,%
   showstringspaces=false,%
   escapeinside={(*@}{@*)}}   
   
\lstset{
 literate={ą}{{\k a}}1
 {Ą}{{\k A}}1
 {ż}{{\. z}}1
 {Ż}{{\. Z}}1
 {ź}{{\' z}}1
 {Ź}{{\' Z}}1
 {ć}{{\' c}}1
 {Ć}{{\' C}}1
 {ę}{{\k e}}1
 {Ę}{{\k E}}1
 {ó}{{\' o}}1
 {Ó}{{\' O}}1
 {ń}{{\' n}}1
 {Ń}{{\' N}}1
 {ś}{{\' s}}1
 {Ś}{{\' S}}1
 {ł}{{\l}}1
 {Ł}{{\L}}1
}   

% --- support for links -----------------------------------------------	
\usepackage{hyperref}
\hypersetup{colorlinks=true,
            linkcolor=black,
            citecolor=darkgray,
            urlcolor=darkgray}
\usepackage{xurl}            
\urlstyle{same}

% --- support for large tables and other stuff ------------------------	
\usepackage{float}
\usepackage{caption}
\usepackage{subcaption}
\usepackage{wrapfig}

% --- support for game theory ------------------------------------------
\usepackage{sgame}

% --- support for no widows --------------------------------------------
\usepackage[defaultlines=4,all]{nowidow}

% -----------------------------------------------------------
\usepackage{setspace}

% --- definitions for environments -------------------------------------
\theoremstyle{definition}
    \newtheorem{condition}{Assumption}
    \newtheorem{example}{Example}      

\theoremstyle{plain}
    \newtheorem{definition}{Definition}    
    \newtheorem{proposition}{Proposition}
    \newtheorem{theorem}{Theorem}
    \newtheorem{cor}{Corollary}

\theoremstyle{remark}
    \newtheorem{remark}{Remark}

% --- other settings --------------------------------------------------
\linespread{1.5}
\frenchspacing
\sloppy
\allowdisplaybreaks[4]
\raggedbottom
\clubpenalty=10000
\widowpenalty=10000

% --- only if required ------------------------------------------------
\AtBeginDocument{\renewcommand*{\figurename}{Figure}}
\AtBeginDocument{\renewcommand*{\tablename}{Table}}

% --- changing definition of footnote ---------------------------------
\makeatletter
\renewcommand\footnotesize{%
   \@setfontsize\footnotesize\@ixpt{10}%
   \abovedisplayskip 8\p@ \@plus2\p@ \@minus4\p@
   \abovedisplayshortskip \z@ \@plus\p@
   \belowdisplayshortskip 4\p@ \@plus2\p@ \@minus2\p@
   \def\@listi{\leftmargin\leftmargini
               \topsep 4\p@ \@plus2\p@ \@minus2\p@
               \parsep 2\p@ \@plus\p@ \@minus\p@
               \itemsep \parsep}%
   \belowdisplayskip \abovedisplayskip
}
\makeatother

% --- useful definitions ----------------------------------------------
\newcommand{\code}[1]{\lstinline{#1}}

% ---------------------------------------------------------------------
\begin{document}

% --- strona tytulowa -------------------------------------------------
\begin{titlepage}
\centering

\includegraphics[width=0.66\textwidth]{logo.JPG}

\vspace*{0.5cm}
Master's study\\
\begin{flushleft}
Field of study: Advanced Analytics – Big Data\\
%Specjalność: <specjalność> % w przypadku braku należy pominać
%Forma studiów: <forma studiów (stacjonarne, itd.)>
\end{flushleft}

\vspace*{.5cm}
\rule{0cm}{1cm}\hfill
\begin{minipage}{9cm}
Author's first name and surname: Bogdan Bojarin\\
Student's register No.: 75184
\end{minipage}

\vspace*{1cm}
\begin{minipage}{12cm}
\centering
\Large
\textbf{Learning and Competition\\in the Differentiated Products Market}
\end{minipage}

\vspace*{2cm}
\rule{0cm}{1cm}\hfill
\begin{minipage}{9cm}
Master's thesis
under the scientific supervision of\\
dr hab. Michał Ramsza\\
written in\\
Institute of Mathematical Economics\\
\end{minipage}

\vfill
Warsaw 2026
\end{titlepage}

\rule{1ex}{0ex}\clearpage

% --- table of contents -----------------------------------------------
\cleardoublepage
\tableofcontents

% --- chapter ---------------------------------------------------------
\cleardoublepage
\section{Introduction}

% Here, we need to introduce the topic, explain why it is interesting, and explain how it connects to theoretical economics. We also need to conduct a literature review.

Firms on real markets need to make pricing and product design decisions in diverse consumer environments with limited information. Consumers have different preferences and willingness to pay for product attributes, and firms tend not to directly see them and cannot accurately forecast demand. Rather, firms use these data over time to direct their decisions. Knowing how companies set prices and attribute products in such contexts is a central question in industrial economics and microeconomics. 

Conventional economic models commonly concern firm behavior with static equilibrium models while using a strong information assumption to establish firm performance. Traditional price competition classic models assume that firms know what is needed and are solving problems analytically in order to reach the equilibrium approach. Although the theoretical contribution of these models is crucial, they lack the insights from two characteristic areas central to many real-world markets: the heterogeneity of consumer preferences and the fact that firms typically learn about demand through experimentation rather than necessarily perfect information. 

This paper constructs some of the components of the structural model of consumer demand in which individual consumers vary in their willingness to pay and prefer product attributes. To compete in a market for distinctively priced goods, companies decide not just prices but also features of the products. Demand is the outcome of individual consumer choices, probabilistically dependent on price and the level of conformity of all features of the product with consumer preferences. Consequently, aggregate demand and firm profits are endogenous results of heterogeneous economic decisions at the micro level. 

Because those profit functions usually do not support analytical solutions, the firms optimization problems are solved numerically. Rather than simply assuming that a firm immediately calculates equilibrium strategies, it employs a simulation-based model in which firms investigate market effects and adapt their alternatives. In both cases, the method provides an opportunity to study market behavior under a limited amount of data, whilst also serving as a bridge between theoretical demand models and algorithmic decision making. To clearly demonstrate how consumer heterogeneity affects pricing, product design and overall, the model is first devised around one firm. The model may be replicated with several firms engaged in a given market in which each firm's customer demand is uncertain in terms of the degree of uncertainty on consumers' demand and the behavior of its competitors. Under such circumstances, equilibrium results are reached by repeated interactions and learning, rather than through explicit decision making. 

The thesis adds to knowledge on price and product competition in differentiated markets by applying a solid demand structure with mathematical and simulation-based solution methods. This approach is firmly rooted in economic theory, although it recognises the reality of features in real-world scenarios like learning, experimenting and incomplete information that cannot be accurately modeled with an analytical approach. For this reason, it offers a flexible basis for studying firm behavior in modern markets characterized by a range of product-based distinctions and consumer heterogeneity, hence product differentiation as a key element of study.

\clearpage
\section{Literature review}

% The place for literature review.

One of the first and most significant works in the field of development and research of agent-oriented models is considered to be the work “Growing Artificial Societies” by D. Epstein and R. Acastell, 1996, in which the authors put forward a new research program based on agent-oriented modeling. Of course, they were not the first to use an agent-based approach in the study of socio-economic phenomena. The first meaningful attempts to apply it can be found in the works of Schelling (1971), Albin and Foley (1990), Axelrod and Bennett (1993), Axelrod (1997), Danielson (1996) and others. However, Epstein and Axtell were the first to explicitly outline the contours of a new methodology. "Our broad goal is to begin developing a unified social science in which evolutionary processes would be embedded in a computational environment that simulates social and economic phenomena. One day, when people ask about the possibility of explaining something, they will mean the ability to grow it in silico, demonstrating how a limited set of microfactors generates macro-phenomena. The development of IT technologies opens up the prospect of looking at the world of social phenomena through the prism of a new science." (\cite{epstein1996})

The research program on artificial societies sees the prospect of overcoming the problem of the subjective component of social processes in the development of computing technologies and agent-oriented modeling methodology. Virtual analogues of the designed systems allow simulation experiments to be conducted and complex hypotheses to be generated and verified. Over the past decade, a large number of works have appeared that claim an agent-oriented approach as their methodological basis.

Despite sharing common fundamental features, including the presence of several autonomously acting entities seeking to maximize an inherent objective function, agent models can vary significantly in terms of the degree of abstraction and detail of the processes being analyzed. In one model, agents can be elementary entities of socio-economic relations, such as individuals (buyers, sellers, voters, road traffic participants) or aggregated entities acting on behalf of several individuals (legal entities, public law entities, unions, coalitions). Differences in the level of abstraction of agent models are determined by the specificity of the represented entities, the quality of the content of their target functions, and the ways in which they interact. There are also models represented by many agents, whose mechanisms and interaction goals are abstract in nature, are complex to parameterize and are focused on studying emerging behavior and emergent effects. Information about the modes of interaction of individual entities allows us to formulate and investigate hypotheses about the global dynamics of systems. However, possessing information about the global dynamics of a complex system allows us to formulate and test hypotheses about the rules of functioning of its constituent agents.

\subsection{High-level abstraction agent modeling}

Highly abstract agent modeling is typically used in studies of objects with a high subjective component. An example of such an object is the marriage market, whose participants, being both commodities and consumers, make their choices based on several subjective criteria, such as age, level of education and wealth, skin color, ethnic origin of the partner, and more. (\cite{tb2003})

The breadth and analytical power of the agent-based approach in analyzing abstract subjects is reflected in examples of its application in studies of racial segregation (Schelling, 1978), the life cycle of firms (\cite{axtell2006}), stock market dynamics (\cite{sbe2018}), cultural convergence (\cite{axelrod1997}) or electoral procedures (\cite{hsm2019}).

Agent-based modeling differs from classical statistical and econometric approaches, which focus on simply calculating coefficients that determine the relationship between the resulting variables and the assumed regressors, in that it aims to reveal the hidden secrets of the phenomenon under study. It is this advantage that makes it indispensable when studying structural effects.

\subsection{Low-level abstraction agent modeling}

Low-level abstraction agent models are usually and most commonly applied in practice. The subjects in them are defined at the level of individuals taken in specific systems of relations. The relative simplicity of quantitative parameterization and the low dispersion of results variation with identical inputs make such models convenient and reliable forecasting tools. The impact of variations in individual elements on the global dynamics of modeled systems can be intuitively predictable. However, including a large number of factors and types of interacting subjects in the analysis significantly reduces the quality of speculative forecasts. This is precisely where the applied significance of models of this type, widely used in demographic studies, consumer choice studies, evacuation models (\cite{ht2000}), transport optimization (\cite{np1995}), pedestrian traffic (\cite{hm1995}) and many more.

\subsection{Criticism and evaluation of the methodology's capabilities}

Critically commenting on the methodology of agent-based modeling, American anthropologist S. Helmreich notes that the cultivation of artificial society structures is a formalization of research biases. \cite{helmreich1998} This observation is not without merit. However, despite the vulnerability of the agent-based approach to such criticism, its solid scientific foundations cannot be denied—any conclusions obtained with its help are fundamentally falsifiable (in the Popperian sense), and therefore open to empirical verification. Today, the methodology of the agent-oriented approach continues to develop actively. More and more publications are appearing in the literature devoted to a critical analysis of its practical application. The high interdisciplinary potential of the methodology hinders the development of universal criteria for the validity of agent models. The requirements for simulation experiments are actively debated. In particular, the issue of the stability (robustness) of results and the framework of quantitative sufficiency of emulation runs is discussed. (\cite{ljs2015})

The growing trend toward agent-based modeling in the study of abstract objects suggests that it will soon be applied to the humanities to issues in philosophy, psychology, and creativity. In the near future, the agent-based approach will be transformed by the rapidly developing field of artificial intelligence. Opportunities are emerging for the construction of highly complex simulators based on highly intelligent agents (\cite{vallacher2017}). We can expect to see the reconstruction of old models based on more sophisticated operating algorithms. The current agenda allows us to predict an increase in the use of the agent-based approach in analyzing epidemics and the consequences of other external economic shock. \cite{ylz2018}

The development of digital technologies has led to a qualitative change in the utilitarian properties of computing technology. Software adaptation of complex cognitive schemes has turned the computer into an innovative tool for solving creative problems. The increase in the computing power of modern computers, the development of new cognitive technologies, and the agent-oriented approach in particular, are bringing humanity closer to the possibility of solving problems of unlimited complexity.


% --- chapter ---------------------------------------------------------
\clearpage
\section{Mathematical models}

Here we give mathematical formulation of all models, but also for every model we numerically solve the model for some selected values of initial and exogenous parameters. We need to make those variables exactly the same as the one we take for simulations so that we can compare the results.

\subsection{A model with a single firm}

Herein, we describe a mathematical model of a market with a single firm. The model's aim is to explain how heterogeneity among consumers creates demand and how a firm will choose prices and product attributes in response to that demand. Since the resulting profit function does not admit a closed-form solution, the decision problem of the firm is analyzed using a numerical, algorithmic approach reflecting learning under incomplete information.

\subsubsection{A model of an agent population}

We consider a market populated by a large number $N$ of agents. Agents are heterogeneous along two dimensions: Reservation price, where each $i$ consumer has a reservation price $R_i > 0$ showing the maximum price at which the consumer would buy the product, and Taste (the ideal product characteristic), where each consumer $i$ has a preferred product characteristic $S_i \in [0,1]$. We assume that the random variables $R$ and $S$ are independent. 

The pair of these random variables is taken from continuous distributions. In the numerical implementation used throughout the thesis, the Gamma distribution was used for the variable $R$ to reflect that there is a certain common reservation price in the population, but that there are some deviations as well. The type of distribution for random variable $S$ is about the particular style preference of an agent. Because we want the values of this distribution to fall in the interval [0,1], we used the Beta distribution. The choice of the Beta distribution was made because it contains the Uniform distribution, but also it may reflect the fact that there is a common value for the style choice with some variation. Therefore, we are ensured that reservation prices are positive and tastes are bounded on [0,1]. These distributions provide a flexible and economically convincing description of consumer heterogeneity.

According to the model, agents make a binary decision: either purchase one unit of the product or do not buy at all. In that way, the first step for the agent is to look at the price of the good. The utility of the agent from facing the good with price \(p\) and style \(o\) is the following
\[
u(r, s, p, o) = -\alpha \frac{p}{r} - \beta (s - o)^2
\]
where $r$ is a realization of the random variable $R$, $s$ is a realization of the random variable $S$, $p$ is the offered price of a good (assumed non-negative) with the style $o$. We can view $U$ as the random variable:
\[
U = -\alpha \frac{p}{R} - \beta (S - o)^2
\]
In the above formula, the parameters r and s describe the consumer (the agent) and the parameters $p$ and $o$ describe the good. Both these parameters are controlled by a firm producing the good.

This specification takes into account such key economic factors as the decrease in the utility as price increases relative to the agent’s wish to pay $r$ and with the squared distance between the consumer’s ideal taste $s$ and the product characteristic $o$.

Agents are assumed to choose with some probability. The probability that an agent with characteristics $(r,s)$ buys the product is given by a logit rule:
\[
P(\text{buy}) = \frac{\exp\!\big(u\big)}{1 + \exp\!\big(u\big)}
\]
And the probability of not buying is given by the following formula:
\[
P(\text{not buy}) = 1 - \frac{\exp\!\big(u\big)}{1 + \exp\!\big(u\big)}
\]
In fact, the above probabilities are random variables. In particular, the probability of buying is a random variable $\mathcal{P}$ that reads:
\[
\mathcal{P} = P(\text{buy})
= \frac{\exp(U)}{1 + \exp(U)}
= \frac{\exp\!\left(-\alpha \frac{p}{R} - \beta (S - o)^2\right)}
{1 + \exp\!\left(-\alpha \frac{p}{R} - \beta (S - o)^2\right)}.
\]
This formulation suggests that buyers who have a greater extent of utility will be more inclined to purchase products, as long as one may still randomize individual behavior. Specifically, the probabilistic choice interpretation serves to model markets with incomplete information and repeated interaction.

\subsubsection{A model of a firm}

A single firm supplies one product to the market. The firm chooses two decision variables: a price $p \geq 0$ and a product characteristic (style) $ s \in [0,1]$.
For convenience, to indicate the expected profit of a single firm offering a good at price $p$ and with the style $o$, we assume that the cost function $c(x)=c \cdot x$, and $c>0$. Therefore, the cost function is linear. The expected profit of a firm is then the following:
\[
\begin{aligned}
	\mathbb{E}(\text{profit})
	&= \mathbb{E}\!\big(p \cdot \text{demand} - c(\text{demand})\big) \\
	&= \mathbb{E}\!\big(p \cdot \text{demand} - c \cdot \text{demand}\big) \\
	&= \mathbb{E}\!\big(p \cdot N \cdot \mathcal{P} - c \cdot N \cdot \mathcal{P}\big) \\
	&= \mathbb{E}\!\big((p - c)\cdot N \cdot \mathcal{P}\big) \\
	&= (p - c)\cdot N \cdot \mathbb{E}(\mathcal{P}),
\end{aligned}
\]
where
\[
\begin{aligned}
	\mathbb{E}(\mathcal{P})
	&= \mathbb{E}\!\left(
	\frac{\exp\!\left(-\alpha \frac{p}{r} - \beta (s - o)^2\right)}
	{1 + \exp\!\left(-\alpha \frac{p}{r} - \beta (s - o)^2\right)}
	\right) \\
	&= \int_{0}^{\infty} \int_{0}^{\infty}
	\frac{\exp\!\left(-\alpha \frac{p}{r} - \beta (s - o)^2\right)}
	{1 + \exp\!\left(-\alpha \frac{p}{r} - \beta (s - o)^2\right)}
	\, f_R(r)\, f_S(s)\, ds\, dr .
\end{aligned}
\]

Thus, the expected profit function generalizes the dilemma faced by a company: raising prices increases profit per unit of output, but reduces demand by decreasing the probability of buying.
Since the expected buying probability is determined by a multivariate integral, the optimization problem cannot be solved analytically. Instead, it is approached numerically using a learning algorithm that simulates the firm's behavior under conditions of incomplete information:
\[
\max_{\substack{p \ge 0 \\ o \in [0,1]}}
\; \Pi(p,o)
= (p - c)\, N\, \mathbb{E}\!\left[P(p,o)\right].
\]

\subsubsection{Algorithm for a firm}

In the model, the firm does not have full knowledge of its demand function, but adopts an algorithmic perspective driven by market outcomes and thus learns from what happens there. The firm follows a repeated five-step procedure:
\begin{enumerate}
	\item Drawing $(R_i, S_i)$ from their respective distributions creates a population of consumers. This population remains fixed throughout the analysis.
	\item The firm selects initial values for price $p$ and product characteristic $o$.
	\item Given $(p,o)$, each consumer makes a purchase decision according to the probabilistic rule defined by the utility function.
	\item The firm observes aggregate demand and computes realized (or expected) profit.
	\item By comparing current profit with profits obtained under small alternative changes, the firm adjusts either price or product characteristic. The firm adopts the change that improves profit and repeats the process.
\end{enumerate}

The algorithm keeps in mind the fact that firms do not solve their optimization problems instantaneously but instead experiment, observe outcomes, and ultimately improve their decisions. In the deterministic setup used in the simulations, the firm estimates expected demand (the sum of individual purchase probabilities), which corresponds to averaging outcomes over many repeated sales periods.

\subsection{A model with two firms}

This subsection extends the single-firm framework to a market with two competing firms. The extension preserves the same structure of consumer heterogeneity and probabilistic choice, but modifies the consumer decision problem: instead of choosing between “buy” and “not buy” from one firm, consumers choose among firm 1, firm 2, and an outside option (no purchase) to allow demand to be split endogenously between firms as a function of both prices and product characteristics.
\par As for the model with single-firm, we consider a market populated by a large number $N$ of agents with the same characteristics, which are the Reservation price $R_i > 0$ and Taste, where each consumer $i$ has a preferred product characteristic $S_i \in [0,1]$ with the same parametric choices for numerical analysis as in the single-firm model, which are Gamma distribution for $R$ and Beta distribution for $S$.

% --- chapter ---------------------------------------------------------
\clearpage
\section{Implementation of mathematical models}

In this chapter we describe how we implemented the above mathematical models. So we do not give here any results, we just describe how we implement a consumer, how we implement a firm, what is an algorithm of the behavior of the firms, and so on and so forth.

% --- chapter ---------------------------------------------------------
\clearpage
\section{Simulation results}

We show the results of simulations for various initial and exogenous parameters. 

\subsection{Simulations for a single firm}

\subsection{Simulations for two firms}

% --- chapter ---------------------------------------------------------
\clearpage
\section{Conclusions}

% --- bibliography ----------------------------------------------------
\clearpage
\printbibliography[heading=subbibliography,nottype=online,title={References}]
\printbibliography[heading=subbibliography,type=online,title={Online references}]

% --- abstract --------------------------------------------------------
\clearpage
\addcontentsline{toc}{section}{List of tables}
\listoftables

% --- abstract --------------------------------------------------------
\clearpage
\addcontentsline{toc}{section}{List of figures}
\listoffigures



% --- abstract --------------------------------------------------------
\clearpage
\addcontentsline{toc}{section}{Streszczenie}
\section*{Streszczenie}

Tutaj zamieszczają Państwo streszczenie pracy. Streszczenie powinno być długości około pół strony.


\end{document}


%%% Local Variables:
%%% mode: latex
%%% TeX-master: t
%%% End:
